\chapter{Definição do problema}
\label{chp:problem}

Na patrulha multiagente um conjunto de agentes homogêneos e autônomos $A = 
{a_{1}, ..., a_{k}}$ é 
utilizado para patrulhar um determinado ambiente. O ambiente de patrulha é uma 
área finita com pontos de interesse e caminhos entre esses pontos por onde os 
agentes podem se locomover, tal ambiente pode ser definido formalmente por um 
grafo não direcionado $G = (N, E)$, onde $N$ é o conjunto de vértices 
representando os pontos de interesse da área de patrulha e $E$ é o conjunto de 
arestas do grafo, o qual representa os caminhos entre os pontos do ambiente. 
Outras possíveis informações estáticas e úteis sobre o ambiente podem ser 
definidas como pesos nos vértices ou arestas do grafo $G$, a distância entre os 
pontos, por exemplo, é uma informação bastante útil e que pode ser definida como 
o peso das arestas. Os pesos dos vértices, por sua vez, podem ser utilizados 
para informar à estratégia de patrulha sobre pontos de maior importância, que 
demandam visitas mais periódicas. 

De acordo com uma estratégia de patrulha $S$ os agentes percorrem o ambiente, 
visitando os pontos de interesse continuamente durante o tempo $T$. Os instantes 
de visitas feitas pelos agentes em cada ponto são utilizados para calcular 
alguma métrica de patrulha $M$. A métrica de patrulha serve como medida de 
desempenho e pode ser calculada de modo a refletir diferentes aplicações e 
objetivos de patrulha.

Neste trabalho estamos interessados em considerar a patrulha multiagente com 
restrições na capacidade de comunicação dos agentes patrulhadores. A comunicação 
entre agentes é definida pela estratégia de patrulha, esta pode ser distribuída, 
onde a comunicação ocorre através de trocas de mensagens diretas entre os 
agentes; ou centralizada, onde uma base central de comunicações é responsável 
por receber todas as mensagens e por fazer a distribuição das mesmas, 
enviando-as para os destinatários. 

A forma de comunicação distribuída é mais suscetível à falhas devido à sua 
natureza dinâmica. Os agentes de patrulha estarão se movendo pelo ambiente 
durante todo o tempo e por muitas vezes poderão se encontrar isolados, fora do 
alcance dos demais agentes. Além disso, tentar usar equipamentos de comunicação 
mais potentes não é exatamente uma solução devido à possíveis limitações de 
consumo de energia do agente. 

Podemos modelar a restrição de comunicação como o raio de alcance máximo do 
equipamento de comunicação utilizado pelos agentes. Como já definido, assumimos 
agentes homogêneos, portanto todos devem possuir o mesmo alcance. Definimos a 
cobertura de comunicação de um agente como um círculo de raio máximo $r$ que se 
move juntamente com o agente e tem como centro a localização atual do mesmo no 
ambiente. Esta cobertura define o alcance de comunicação de cada agente durante 
a execução da patrulha onde, um agente fora da cobertura de outros está 
totalmente incapacitado de se comunicar com os demais; para agentes que se 
encontram dentro da área de cobertura de outros agentes a chance da comunicação 
acontecer com sucesso é inversamente proporcional à distância que os agentes se 
encontram um do outro.

